\subsection{База данных в микросервисах}
Для некоторых микросервиов база данных совсем не нужна.
Такие микросервисы обычно не используют никакую информацию, например, сервисы по конвертированию данных.
Там запускается какой-то разовый алгоритм и все.
Часто сохранять информацию в один csv файл.
Однако сервисы, которые в основе своей работают с данными пользователями, просто необходим.
Собранную информацию потом возможно использовать для анализа, чтобы, например, делать на ее основе модели искусственного интеллекта.

Одной из основных причин использования баз данных в микросервисах является возможность распределения данных и обработки запросов между несколькими сервисами.
Каждый сервис может иметь свою собственную базу данных или использовать общую базу данных, что позволяет сократить нагрузку на каждый сервис и повысить производительность всей системы.

Кроме того, базы данных обеспечивают целостность данных и защиту от потери информации. 
Они позволяют хранить информацию в структурированном виде, что упрощает ее поиск и обработку.
Базы данных также обеспечивают безопасность данных, позволяя контролировать доступ к ним и устанавливать права доступа для каждого сервиса.

Базы данных также позволяют решать проблемы масштабирования и управления ресурсами. 
Благодаря возможности горизонтального масштабирования баз данных, можно легко расширять хранение данных и увеличивать производительность системы при необходимости. 
Кроме того, базы данных позволяют эффективно управлять ресурсами, например, памятью и дисковым пространством, что повышает эффективность работы всей системы.

Использование баз данных в микросервисах также упрощает разработку и тестирование сервисов. 
Каждый сервис может использовать свою собственную базу данных для тестирования и разработки, что позволяет избежать конфликтов между сервисами и обеспечить более быстрое и эффективное тестирование.

Также, базы данных помогают поддерживать констистентность данных. Очень важно эту особенность учитывать и в сервисе по поиску картинок.
На ее основе можно сохранять ранее полученную информацию и не беспокоить сайт при каждом новом запросе.
Эта особенность очень важна в тех случаях, если сайт имеет защиту против DDOS атак.
Если сервисом будет пользоваться даже сотни людей, то у сайта может возникнуть вопрос о появлении стольких запросов с одного IP.
Именно поэтому намного лучше и проще будет сохранять всю информацию в базе данных.

Приведу пример, есть популярная манга, которая так или иначе пользуется спросом.
При помощи поиска этой же манги и конкретной главы возможно сразу из памяти БД достать информацию, лишний раз не нагружая сайт запросами.
Таким образом такое поведение очень хорошо скажется на:
\begin{itemize}
    \item работоспособности приложения. Не придется каждый раз ждать ответа от конкретного сайта,
    \item внезапно не будет заблокирован наш микросервис за большое количество запросов на сайт,
    \item не придется в микросервисе обходить защиту от роботов (captcha).
\end{itemize} 

Перед тем, как использовать БД, нужно также подумать о том что будет храниться там. Очень важно это сделать на этапе конструирования.
Иначе может получиться так, что эффекта от такого мощного инструмента не будет.

Также при помощи БД возможно сделать подписку на обновление контента. С ее помощью можно организовать рассылку, достав большой объем нужных данных.
С файлом csv это было бы очень трудно сделать.

Также преимуществом работы с БД является транзакции.
При помощи транзакции можно изменять чувствительные данные, не боясь при этом гонки (race condition).