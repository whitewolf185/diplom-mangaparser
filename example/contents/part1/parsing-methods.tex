\subsection{Методика парсинга}
Понимание сайта --- важная часть абсолютно любого парсинга данных. Как понятно из определения, необходимо собирать информацию.
Для того, чтобы собирать информацию, необходимо знать как она расположена.
Для того, чтобы понимать как расположена информация, необходимо исселодовать пути появления этой информации перед пользователем и 
выяснить откуда она берется.

Плавно мы подбираемся к тому, что знание устройство сайта, клиентской части и сервера --- необходимы для ранее сказанного парсинга данных.
Другими словами, для парсинга данных с сайта необходимо полностью понимать как тот устроен. 
Если не получается полностью изучить, то нужно понять что за данные нужны, чтобы в конечном итоге построить то, что нужно.
Нужно цепляться буквально за любую крупицу данных, которая может так или иначе помочь.

\subsubsection{Взаимодействие с сервером}
Для поиска подобной крупицы информации нужно понимать, что ничего не происходит из ниоткуда. Везде есть какие-то следы.
Они могут быть зашифрованы, или спрятаны за тонной других запросов. Но информация откуда-то получается.
Нужно сделать оговорку, потому что есть сайты, которые не используют сервер для получения какой-либо информации, 
а она уже сразу закодирована в верстке сайта. Такими сайтами называются <<визиткой>>.
Как понятно из навзвания, такие сайты нужны для того, чтобы красиво продемонстрировать род деятильности, прорекламировать компанию,
которую представляет тот или иной сотрудник и указать там контакты для сотрудничества.
Обычно такие сайты одностроничные, под собой не имеют никакой логики. Такие сайты очень удобно парсить.
Но чаще всего человек сталкивается с сложными сайтами со сложной логикой, где есть большая база пользователей и контета.
Такие сайты просто обязаны обращаться за какой-то логикой на сервер.

Можно заметить, что в разделе\ref{js-ref} упоминается JS --- логическая сторона сайта. Почему не пользоваться этой замечательной компонетой сайта для различной логики?
Дело, конечно же, в безопасности. Куда более надежно будет положить какую-то информацию в переменную, а потом ее в будущем отобразить на странице при помощи, например, jQuery.
Никто не хочет, чтобы алгоритмы обработки личной информации лежали перед всеми на видном месте, чтобы их можно было запустить и все расшифровать.
Или как еще по-другому можно хранить миллионы миллиардов информации?

Итак, так как предмет исследования как раз идет за нужной информацией на сервер, 
необходимо понять из каких крупиц информации создаются следующие более сложные запросы на сервер.