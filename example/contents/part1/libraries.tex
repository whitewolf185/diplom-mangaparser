\subsection{Обоснование библиотек}

В микросервисе были использованы внешние модули для упрощения частоиспользуемых действий в коде.
Каждая из них представляет собой очень полезную функциональность, которая будет перечислена ниже с обоснованием использования.

\subsubsection{pdfcpu}
Pdfcpu \cite{pdfcpu-cite} --- пакет, написанный на ЯП GO, яляющийся мощным сборником pdf файлов. Представляет интферфейс для создания сложноструктурированных pdf файлов.
В данном пакете возможно создавать абсолютно любые pdf файлы с различными метаданными документа, начиная с содержания, заканчивая указыванием автора написания документа.
Библиотека способна также использовать различные картинки различных разрешений, позволяя менять их размер, цвет, прозрачность и другие параметры.
Пакет был взят в использование потому, что предоставляется разработчиками удобный и понятный интферфейс для собирания большого количество картинок в pdf файл. 
Также модуль отличается своей быстротой исполнения поставленных задач. Небольшая выдержка кода может продемонстрировать простоту использования пакета.

\begin{figure}
\begin{lstlisting}[language=go]
func CreatePDFFromImagesDir(imagesDirPath string, outputPath string) error {
	imagesPath, err := GetImagesPathStr(imagesDirPath)
	if err != nil {
		return errors.Wrap(err, "something wrong with creating images path")
	}
	err = api.ImportImagesFile(imagesPath, outputPath, nil, nil)
	if err != nil {
		return errors.Wrap(err, "pdf creation filure")
	}
	return nil
}
\end{lstlisting}
\caption{Демонстрация простоты интерфейсы pdfcpu}
\label{libs-pdfcpu}
\end{figure}