\subsection{Обоснование библиотек}

В микросервисе были использованы внешние модули для упрощения частоиспользуемых действий в коде.
Каждая из них представляет собой очень полезную функциональность, которая будет перечислена ниже с обоснованием использования.

\subsubsection{Pdfcpu}
Pdfcpu \cite{pdfcpu-cite} --- пакет, написанный на ЯП GO, яляющийся мощным сборником pdf файлов. Представляет интферфейс для создания сложноструктурированных pdf файлов.

В данном пакете возможно создавать абсолютно любые pdf файлы с различными метаданными документа, начиная с содержания, заканчивая указыванием автора написания документа.
Библиотека способна также использовать различные картинки различных разрешений, позволяя менять их размер, цвет, прозрачность и другие параметры.

Пакет был взят в использование потому, что предоставляется разработчиками удобный и понятный интферфейс для собирания большого количество картинок в pdf файл. 
Также модуль отличается своей быстротой исполнения поставленных задач. Рисунок~\ref{libs-pdfcpu} может продемонстрировать простоту использования пакета.

\begin{figure}
\begin{lstlisting}[language=go]
func CreatePDFFromImagesDir(imagesDirPath string, outputPath string) error {
	imagesPath, err := GetImagesPathStr(imagesDirPath)
	if err != nil {
		return errors.Wrap(err, "something wrong with creating images path")
	}
	err = api.ImportImagesFile(imagesPath, outputPath, nil, nil)
	if err != nil {
		return errors.Wrap(err, "pdf creation filure")
	}
	return nil
}
\end{lstlisting}
\caption{Демонстрация работы интерфейса pdfcpu}
\label{libs-pdfcpu}
\end{figure}


\subsubsection{Goquery}
Goquery \cite{goquery-cite} --- пакет, написанный на ЯП GO, который способен выполнять задачи библиотеки jQuery \cite{jquery-cite}. 

Данный пакет используется для того, чтобы парсить DOM дерево полученного HTML файла. С ним возможно быстро и удобно получить нужную информацию с любого сайта.
После реверс-инженеринга, благодаря этому пакету, быстро и просто достается информация о картинках, классифицируется на разделы и подразделы (например, группировка на главы).

\begin{figure}
	\begin{lstlisting}[language=go]
	func (mlb mangaLibController) getChapterID(doc *goquery.Document) (string, error) {
		result, ok := doc.Find("#comments").Attr("data-post-id")
		if !ok {
			return "", errors.Wrap(customerrors.ErrEmptyAttr, "mangalib: data-post-id")
		}
		return result, nil
	}
	\end{lstlisting}
	\caption{Демонстрация работы пакета goquery}
	\label{libs-goquery}
\end{figure}

Как видно из рисунка~\ref{libs-goquery}, используются команды jQuery для получения информации о главе. 
Впоследствии эта глава используется для того, чтобы дастать из неё картинки.

\subsubsection{Jet} \label{jet-section}
Go-jet \cite{jet-cite} --- пакет, написанный на ЯП GO, являющийся конструктором SQL запросов в БД.

С помощью этой библиотеки возможно собирать любые SQL запросы, избегая уязвимость --- SQL-инъекцию. 
Этот пакет подключается к БД и генерирует код на языке GO, при помощи которого можно собирать различные SQL запросы.
От инъекции помогает избавиться тот факт, что абсолютно весь запрос делается при помощи конструктора, в процессе сбора которого есть постоянные проверки на соответствие данных,
поступающих в процесс создания запроса.
Также стоит заметить, что сама возможность собирать запрос из различных кусочков, может очень сильно помочь при составлении денамического запроса, который так или иначе зависит от поступивших на вход параметров.

Именно по этой причине был выбран данный пакет.
Как можно видеть из рисунка~\ref{libs-jet}, собирается SQL запрос в зависимости от пришедших на вход параметров. 
Таким образом, условная секция запроса может варьироваться.

\begin{figure}
	\begin{lstlisting}[language=go]
		func (perC personController) GetEmailByID(ctx context.Context, person domain.PersonInfo) (string, error){
			selectStmt := table.Persons.SELECT(table.Persons.Email)
			var whereStmt postgres.BoolExpression
		
			personID, _ := uuid.Parse(person.PersonID)
			switch {
			case personID != uuid.UUID{} && personID != uuid.Nil:
				whereStmt = table.Persons.ID.EQ(postgres.UUID(personID))
			case person.TelegramID > 0:
				whereStmt = table.Persons.TelegramID.EQ(postgres.Int64(person.TelegramID))
			default:
				return "", customerrors.ErrEmailsNotFound
			}
			stmt, args := selectStmt.WHERE(whereStmt).Sql()
		
			...
		
			return email, nil
		}
	\end{lstlisting}
	\caption{Демонстрация работы пакета jet}
	\label{libs-jet}
\end{figure}

