\subsection{Устройство сайтов}
Предметом исследования диплома был ограниченный список сайтов,
участвующих в сборе картинок с последующей их сортировкой и группировкой в pdf файл для последующего их удобного просмотра или прочтения.
Суть изучения состояло в том, чтобы произвести реверс-инженеринг сайта, понять как он работает. 
Затем использовать полученные знания в поиске нужной мне информации, путем парсинга страницы сайта, разбиения полученной структуры сайта на составляющие,
с последующей структуризацией в нужный мне формат для правильной отдачи итогового результата.

Абсолютно любой сайт состоит из 3-х компонент:

\begin{itemize}
    \item HTML,
    \item CSS,
    \item JavaScript.
\end{itemize}  

\subsubsection{HTML}
В данной компоненте описывается различная информация о странице. Будь то текст статьи, картинки и прочего.
Затем, браузер, применяя свои движки визуализации, преображает инфорацию, содержащуюся в компоненте, в понятное для человека визуализированное представление.
HTML имеет свою структуру, называемой DOM деревом. Эта структура по сути своей является набором различных тэгов.
Для идеального примера можно сказать, что HTML похож на LATEX. Таким образом, HTML --- это то, откуда будет получаться нужная информация.

Как было сказано ранее, HTML состоит из тэгов. Есть следующие виды тэгов:
\begin{itemize}
    \item двойные,
    \item ординарные.
\end{itemize} 

Двойной тэг отличается от ординарного тем, что у двойного тэга есть закрывающий тэг. Между открывающим тэгом и закрывающим пишется текстовая информация.
Например, на рисунке~\ref{html-tag-2} продемонстрирован двойной тэг, обозначающий параграф. Обычно такой тэг используется в различных статьях.
На рисунке~\ref{html-tag-1} продемонстрирован пример ординарного тэга, обозначающий картинку. Когда браузер видит этот тэг, он понимает, что перед ним картинка.
За кадром происходит запрос на сервер по ссылке, расположенной в мета информации, с последующим отображением изображения.

\begin{figure}
    \begin{lstlisting}[language=html]
        <p class="paragraf">Текст параграфа</p>
    \end{lstlisting}
    \caption{Пример двойного тэга}
    \label{html-tag-2}
\end{figure}
\begin{figure}
    \begin{lstlisting}[language=html]
        <img src="picture.img" alt="Cup of Kitty">
    \end{lstlisting}
    \caption{Пример ординарного тэга}
    \label{html-tag-1}
\end{figure}

На рисунках \ref{html-tag-2} и \ref{html-tag-1} были продемонстрированы тэги, которые требуют от браузера некоторых действий.
Но, как было выше отмечено, HTML очень похож на LATEX. И похож он тем, что в основном страницы сайта нужно верстать. 
Поэтому люди придумали тэг, который не требует никаких действий со стороны браузера, но при этом будет как-то объединять информацию.

На рисунке~\ref{html-div} изображен тэг div. Этот тэг нужен исключительно для того, чтобы как-то объединять информацию, присваивая ей какую-то общую метаинформацию, например, класс.

\begin{figure}
	\begin{lstlisting}[language=html]
	<div class="pritty"></div>
	\end{lstlisting}
	\caption{Тэг div}
	\label{html-div}
\end{figure}

Из совокупности тэгов состоит страница сайта. А структура, которая в конечном итоге образуется, называется DOM деревом.

\subsubsection{CSS}
Эта компонента, в свою очередь, отвечает за то как будет выглядить информация, содержащаяся в HTML. По сути своей, CSS --- оформление страницы.
После появления данной компоненты, сайты в интернете стали выглядить так, как мы их сейчас видим.

Для формирования различных стилей, используется метаинформация тэгов. 
Удобнее всего для таких нужд оказалось использовать тэг, изображенный на рисунке~\ref{html-div}, так как этот тэг как никто лучше служит для объединения информации, которую можно как-то по-особенному
расположить и украсить.

Именно поэтому на большинстве сайтов можно увидеть именно этот тэг.

\subsubsection{JavaScript} \label{js-ref}
Эта компонента отвечает за логику на страницах. Логика может использоваться при анимировании сайтов и общении с сервером. Общение с сервером нужно для того, чтобы возможно было у него получить какую-то информацию.


 
