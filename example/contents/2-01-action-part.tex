\section{ПРАКТИЧЕСКОЕ ПРИМЕНЕНИЕ}

\subsection{Структура приложения}
Приложение состоит из основной серверной части --- API и клиентских частей. 
Клиентским сервисом считается любой сервис, например бот в мессенджере, способный отправлять запросы на серверную часть.
Серверная часть выполнена в соответствии с концепцией чистой архитектуре, в соотвествии с тем, что описан в рзеделе \ref{clean-arch}.
На рисунке~\ref{clean-arch-add} изображена схема проекта.
Каждый пакет отвечает за свою определенную часть.

Пакет <<repository>> отвечает за взаимодействие между БД. В <<cmd>> находится main.go файл, который запускает сервер.
<<App>> содержит в себе реализацию хендлер функций. Пакет <<config>> используется по всему проекту. 
В нем описаны функции, помогающие доставать различные секреты приложения, такие как пароли, из переменных окружения.
В этом же пакете происходит парсинг тех из yaml файла.
В пакете <<gen>> содержится сгенерированные структуры библиотекой jet из раздела \ref{jet-section}. 
В <<internal/pgk>> содержатся все пакеты, которые так или иначе выполняют какую-то логигу приложения.
Своего рода usecase слой.
Пакет <<parse>> отвечает за логику парсинга описанного для конерктного сайта. Так можно увидеть там один из объектов исследования --- сайт mangalib
<<Pdf\_creator>>, как понятно из названия, занимается описанием логики сбора картинок в один pdf файл.
Пакет <<mailer>> необходим для того, чтобы отправлять файлы через через какие-либо возможные способы передачи, например, отправки тех через электронную почту.
В папке <<migrations>> описан SQL код всех миграций БД.

Внешние пакеты (те, что находится не в папке <<internal>>) используются для того, чтобы к ним возможно было получить доступ мнешним клиентским сервисам.
Так в <<api>> пакете описаны структуры request и response, отвечающие за боди запроса и ответа соответственно.
Также в модуле описан роутинг API. 
Пакет <<pgk>> необходим для того, чтобы импортировать тот в клиентский сервис, так как там предполагается описание sdk.
В этом пакете описаны кастомные ошибки, которые можно в последствии обработать.