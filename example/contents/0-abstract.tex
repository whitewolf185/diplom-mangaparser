\abstract % Структурный элемент: РЕФЕРАТ

\keywords{МАНГА, БАЗА ДАННЫХ, МИКРОСЕРВИС, ПАРСИНГ, САЙТ, API}

Работа состоит из введения, двух глав и заключения. 
Темой работы является веб приложение, основанное на технологии парсинга данных для сбора тематических картинок (манги) и компилирование их в сборники в одном pdf файле.

Целью работы является написание API, которое возможно использовать в других клиентских сервисах, таких как мессенджер, сайт.
API должен предоставлять интерфейс для отправки собранных изображений на электронную книгу через отправку тех по электронной почте.

Проведение работы состоит из исследования целевых сайтов, список которых задается заранее. 
Написание парсинг алгоритма. Конструирование микросервиса, применяя методику чистой архитектуры для дальнейшего разввития проекта.

Областью применения является любой клиентский сервис, способный общаться с сделанным API.

Итог внедрения считается успешно работающее API любым из возможных клиентских сервисов, таких как сайт, мессенджер.

Дальнейшее развитие проекта предполагает оформление подписки на сайты для автоматического информирования пользователя о новых главах комиксах или манги.
Оптимизирование работы сервиса для успешного обслуживания больше десятка пользователей.

Появление подобного микросервиса должно принести в рынок Open source новое кроссплатформенное решение по оффлайн прочтению манги, в том числе, и на электронных устройствах с искусственными чернилами (электронные книги).