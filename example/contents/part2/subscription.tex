\subsection{Подписка на картинки}
Было бы очень удобно пользователем автоматически получать обновление на свою электронную книгу своей любимой манги.
Поэтому было придумано сделать подписку на определенную мангу на сайте, чтобы можно было не взаимодействуя с API, получить нужную главу манги, если та вышла.
Подписка необходима для тех людей, которые имеют огромное количество манги и им просто сложно бывает уследить за тем, что они читали.

Используя все ранее сделанное, мы можем организовать рассылку при помощи базы данных и собранной информацией о главах манги.
Так у нас хранится информация в таблицах о структуре манги и есть список пользователей, которые так или иначе захотели подписаться на определенную мангу.

Так как процесс обхода каждой манги необходимо делать без дополнительной помощи со стороны пользователя, было принято решение о том, чтобы запустить некий воркер, способный проходить по всем мангам, которые есть в таблице и обновлять эту структуру.
Однако, сразу может прийти в голову мысль о том, что имея при себе 1000 пользователей, которые подписались на 1000 или более комиксов, нам придется сделать не меньше 1000 запросов на сайты с мангой, чтобы обновить структуру глав и отследить обновление.
Поэтому нам нужно каким-то образом уменьшить нагрузку на сайт, чтобы нас случайно не заблокировали.

Была придумана схема которая позволяло бы смотреть время последнего обновления структуры манги и задавать не одинаковое время проверки структуры главы.
Есть смысл обновлять структуру не чаще раза в сутки, так как новая глава выходит, самое быстрое, раз в неделю.

Есть более сложный способ, который предполагает смотреть на статистику выхода главы и динамически задавать диапазон проверки на выход следующей главы через примерно похожий промежуток времени.
Например, есть комиксы, главы которых выходят не чаще, чем раз в месяц. Поэтому смысла проверять обновления структуры глав совершенно не стоит.

Для всех вышеизложенных идей нам поможет собранная информация о мангах в таблицах БД.
Соответсвенно в таблицу об информации о манге добавится еще одно значение --- время, после которого стоит проверить структуру глав на обновление.

Далее должен присутствовать воркер, который раз в сутки, а можно и даже в разы чаще, проверять время в таблице с информацией о манге.
Так, если мы заметим, что в столбце со временем значение окажется меньше, чем настоящее время, мы пойдем делать запросы на обновления структуры глав комикса.

Проблема, которая может возникнуть внезапно, что комиксы могут выходить в одно время, но вероятность этого крайне мала, а несколько десятков запросов сайт сможет выдержать.