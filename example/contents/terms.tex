\newglossaryentry{html}{ % Нужны разные id, можно ставить просто последовательно
    name={HTML},
    description={стандартизированный язык гипертекстовой разметки документов для просмотра веб-страниц в браузере.} 
}
\newglossaryentry{http}{ % Нужны разные id, можно ставить просто последовательно
    name={HTTP},
    description={протокол передачи гипертекста, изначально в виде гипертекстовых документов в формате HTML, в настоящее время используется для передачи произвольных данных.} 
}
\newglossaryentry{dom}{
    name={DOM-дерево},
    description={структурное представление HTML файла.} 
}
\newglossaryentry{api}{
    name={API},
    description={описание способов взаимодействия одной компьютерной программы с другими.} 
}
\newglossaryentry{method}{
    name={Метод HTTP},
    description={последовательность из любых символов, кроме управляющих и разделителей, указывающая на основную операцию над ресурсом. Обычно метод представляет собой короткое английское слово, записанное заглавными буквами.}
}
\newglossaryentry{request}{
    name={HTTP запрос},
    description={сигнал серверу через http метод, всегда требующий какого-то ответа от сервара.}
}
\newglossaryentry{handler}{
    name={Хендлер},
    description={функция, которую выполняет веб-приложение, когда был сделан соответствующий http запрос.}
}
\newglossaryentry{bd}{
    name={База данных},
    description={набор информации, которая хранится упорядоченно в электронном виде.}
}
\newglossaryentry{server}{
    name={Сервер},
    description={сервером в веб приложении называют программу, которая занимается обработкой запросов с клиентской части приложения. В основном на сервере происходит вся логика работы веб приложения. Обычно на серверной части запускается база данных.}
}
\newglossaryentry{client}{
    name={Клиент},
    description={приложение, которое путем отправки запросов на сервер способно визуализировать полученную информацию. Обычно приложение клиент общается с пользователем. При этом не присутствует никакой логики, связанной с основным приложением. Например, браузер является клиентом.}
}
\newglossaryentry{lib}{
    name={Библиотека},
    description={сборник подпрограмм или объектов, используемых для разработки программного обеспечения.}
}
\newglossaryentry{parsing}{
    name={Парсинг},
    description={это автоматизированный сбор и структурирование информации с сайтов при помощи программы или сервиса.}
}
\newglossaryentry{jquery}{
    name={jQuery},
    description={набор функций JavaScript, фокусирующийся на взаимодействии JavaScript и HTML. Библиотека jQuery помогает легко получать доступ к любому элементу DOM, обращаться к атрибутам и содержимому элементов DOM, манипулировать ими.}
}
\newglossaryentry{consistency}{
    name={Конститентность},
    description={согласованность данных друг с другом, целостность данных, а также внутренняя непротиворечивость.}
}
\newglossaryentry{migration}{
    name={Миграция базы данных},
    description={переход от одной структуры базы данных к другой без потери косистентности.}
}
\newglossaryentry{pipeline}{
    name={Пайплайн},
    description={конвеер данных, поступающий из одной логической программы другой.}
}
\newglossaryentry{cleanarch}{
    name={Чистая архитектура},
    description={понятие при конструировании микросервиса, созданное для разделения различных концепций, путем написания кода на нескольких уровнях с четким правилом, которое позволяет создать тестируемый и поддерживаемый проект.}
}
\newglossaryentry{golang}{
    name={GO},
    description={язык программирования, созданный компанией гугл преимущественно для создания бекэнд сервисов.}
}
\newglossaryentry{module}{
    name={Модуль},
    description={называется пакет в проекте языка GO, который может быть переиспользован в других приложениях или проектах, как внешняя библиотека. Был добавлен в версии 1.11.}
}
\newglossaryentry{pkg}{
    name={Пакет},
    description={понятие в языке GO, обозначающее коллекцию исходного кода в одной дериктории скомпилируемых вместе. Пакет также является подключаемым модулем.}
}
\newglossaryentry{reverse}{
    name={Реверс-инженеринг},
    description={исследование некоторого готового устройства или программы, а также документации на него с целью понять принцип его работы. Также известно, как <<обратная разработка>>.}
}
\newglossaryentry{sql-injection}{
    name={SQL-инъекция},
    description={один из распространённых способов взлома сайтов и программ, работающих с базами данных, основанный на внедрении в запрос произвольного SQL-кода.}
}
\newglossaryentry{ddos-attack}{
    name={Dos-атака},
    description={ хакерская атака на вычислительную систему с целью довести её до отказа, то есть создание таких условий, при которых добросовестные пользователи системы не смогут получить доступ к предоставляемым системным ресурсам (серверам), либо этот доступ будет затруднён. Если атака выполняется одновременно с большого числа компьютеров, говорят о DDoS-атаке}
}
\newglossaryentry{manga}{
    name={Манга},
    description={японские комиксы, иногда называемые комикку.}
}
\newglossaryentry{transaction}{
    name={Транзация},
    description={это набор операций по работе с базой данных, объединенных в одну атомарную пачку.}
}